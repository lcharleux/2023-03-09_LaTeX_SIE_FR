\documentclass[10pt,a4paper]{article}
% PACKAGES
\usepackage[utf8]{inputenc}
\usepackage[francais]{babel}
\usepackage[T1]{fontenc}

\usepackage{amsmath}
\usepackage{amsfonts}
\usepackage{amssymb}
\usepackage{hyperref}
\hypersetup{
    colorlinks=true,
    linkcolor=blue,
    filecolor=magenta,      
    urlcolor=cyan,
    pdftitle={Un document vraiment indispensable},
    pdfauthor={ED SIE !!!},
    pdfpagemode=FullScreen,
    }

\usepackage[left=2cm,right=2cm,top=2cm,bottom=2cm]{geometry}

\usepackage{lipsum}

%SETUP
\title{Mon premier document \LaTeX}
\author{ED SIE}

% DOCUMENT
\begin{document}

\maketitle
\tableofcontents

\section{Introduction}
Du texte 
\subsection{Une première sous section}
Du texte 
\subsection{Une autre sous section}
Du texte 

%\lipsum % Du texte pou meubler


\section{Environnements utiles}
\label{sec:environnements}

Ceci est une liste à puces:

\begin{itemize}
	\item Une chose.
	\item Une autre chose.
\end{itemize}

Une énumération:

\begin{enumerate}
	\item Une chose.
	\item Une autre chose.
\end{enumerate}

%\lipsum % Du texte pou meubler

\section{Les maths}
Après avoir parlé des environnements dans la partie \ref{sec:environnements}. On décide maintenant de parler des mathématiques et de leur écriture dans \LaTeX. On peut écrire des équations:

$$
u_{ij}(x) = \int_0^\infty \alpha_{ij}(t,x) dt
$$

On peut aussi écrire des maths en ligne, par exemple $\alpha = 5$ est un environnement mathématique en ligne. On peut aussi numéroter les équations comme dans l'indispensable équation \ref{eq:awesome_equation} ci-dessous:

\begin{equation}
	U = \sum_{i=0}^N \dfrac{a_i}{b_i}
	\label{eq:awesome_equation}
\end{equation}

Il existe d'autres environnements de maths, par exemple:

\begin{align}
	f(x) & = (x+2)(x+4)   \\
	     & = x^2 + 6x + 8 
\end{align}

\begin{equation}
	g(x) = \left\lbrace 
	\begin{split}
		5 \mbox{ si } x \in ]-\infty, 5] \\
		0 \mbox{ otherwise }	
	\end{split}	
	\right.
\end{equation}

\end{document}