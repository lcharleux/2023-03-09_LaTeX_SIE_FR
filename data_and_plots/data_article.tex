\documentclass{article}
\usepackage[utf8]{inputenc}
\usepackage[francais]{babel}
\usepackage[T1]{fontenc}

\usepackage{amsmath}
\usepackage{amsfonts}
\usepackage{amssymb}
\usepackage{hyperref}
\hypersetup{
    colorlinks=true,
    linkcolor=blue,
    filecolor=magenta,      
    urlcolor=cyan,
    pdftitle={Un document vraiment indispensable},
    pdfauthor={ED SIE !!!},
    pdfpagemode=FullScreen,
    }

\usepackage[left=2cm,right=2cm,top=2cm,bottom=2cm]{geometry}
\usepackage{booktabs}
\usepackage{graphicx}
\usepackage{tikz}
\usepackage{pgfplots}


\title{Import data into \LaTeX}
\author{SIE}
\begin{document}
\maketitle
\tableofcontents

\section{Tabular data}

\begin{table}
	\begin{center}
		\input{data/data_table.tex}
	\end{center}
	\caption{Data from Python.}
\end{table}


\section{Figure associée}





\begin{figure}[h!]
	\begin{center}
		\begin{tikzpicture}
			\begin{axis}[
					width=\linewidth, % Scale the plot to \linewidth
					grid=major, 
					grid style={dashed,gray!30},
					xlabel= Time $t$, % Set the labels
					ylabel= Acceleration $a$,			          
					legend style={at={(0.5,-0.2)},anchor=north},
					x tick label style={rotate=0,anchor=north}
				]
				\addplot 
				% add a plot from table; you select the columns by using the actual name in
				% the .csv file (on top)
				table[x=t,y=a,col sep=comma] {data/data_full.csv}; 
				\legend{Data from Python}
			\end{axis}
		\end{tikzpicture}
		\caption{My first autogenerated plot.}
	\end{center}
\end{figure}

\end{document}